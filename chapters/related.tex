\chapter{Related work}
In this chapter we revise existing normal estimation approaches and justify the choice of geometrical method on top of which we build our pipeline.

\section{Plane fitting}
In most cases normals are calculated using plane fitting of a point neighbourhood \cite{normal_est}. Given a set of 3d points $P = \left[\bm{p}_1 \dots \bm{p}_k\right]$ one should solve a task of finding best fitting plane. Plane is represented using normal vector $\bm{n}$ and distance to origin $d$. Below several existing approaches are revised.

\subsection{Centered SVD} \label{center-svd}
Most popular estimation method consist of centering point matrix and then solving following minimization task:
\[
\min_{\bm{n}} \| \left(P^\top - \bar{P}^\top \right) \bm{n} \|_2.
\]
As showed in \cite{surface_rec} this can be done by applying SVD on $3 \times 3$ covariance matrix
\[
C = \left(P - \bar{P}\right) \left(P^\top - \bar{P}^\top \right)
\]
and then setting $\bm{n}$ to singular vector corresponding to the smallest singular value. It is also worth mentioning that this method is equivalent to applying PCA on matrix $P$ and then choosing component with the least variance. Because of this some researchers may refer to this method as \textit{PlanePCA} \cite{normal_est}.

Described method, however, suffers from necessity to recalculate data cenetering each time new point is added to data matrix.

\subsection{Homogeneous SVD} \label{homog-svd}
Another approach exists which allows to calculate plane parameters without centering of data points.
Following minimization task is solved:
\[
\min_{\bm{b}} \| \left[P^\top \mathbbm{1}_k \right] \bm{b} \|_2,
\]
where $\bm{b}$ stands for $\begin{psmallmatrix}
\bm{n} \\
d
\end{psmallmatrix}$. Similarly to section \ref{center-svd} one can solve this task using SVD decomposition of $4 \times 4$ matrix
\[
C = \left[P^\top \mathbbm{1}_k \right]^\top \left[P^\top \mathbbm{1}_k \right].
\]
As soon as resulting vector $\bm{b}$ is unitary, one should afterwards rescale vector $\bm{n}$ to meet requirement $\|\bm{n}\|_2 = 1$. Oppositely to \textit{Centered SVD} this method supports incremental updates meaning no recentering of data needed; we show how one can effectively use this property in section \ref{integ-im}.


\section{Deep neural network approaches}

Several attempts were taken to estimate normals using deep neural network approaches. They include work "Deep Surface Normal Estimation with Hierarchical RGB-D Fusion", where researchers tried to combine rgb and depth inputs to 