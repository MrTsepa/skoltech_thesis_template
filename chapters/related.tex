\chapter{Related work}
In this chapter, we revise existing normal estimation approaches and justify the choice of the geometrical method on top of which we build our pipeline.

\section{Plane fitting}
In most cases, normals are calculated through plane fitting of a point neighborhood \cite{normal_est}. Given a set of 3d points $P = \left[\bm{p}_1 \dots \bm{p}_k\right]$, one should solve the task of finding best-fitting plane. The plane is represented by normal vector $\bm{n}$ and distance to the origin $d$. Several existing approaches are revised below.

\subsection{Centered SVD} \label{center-svd}
Most popular estimation method consists of centering point matrix and then solving following minimization task:
\[
\min_{\bm{n}} \| \left(P^\top - \bar{P}^\top \right) \bm{n} \|_2.
\]
As showed in \cite{ferrer, surface_rec} this can be done by calculating SVD of $3 \times 3$ covariance matrix
\[
C = \left(P - \bar{P}\right) \left(P^\top - \bar{P}^\top \right)
\]
and then setting $\bm{n}$ to singular vector corresponding to the smallest singular value. It is also worth mentioning that this method is equivalent to applying PCA on the matrix $P$ and then choosing the component with the least variance. Thus, this algorithm may also be referred as \textit{PlanePCA} \cite{normal_est}.

However, the described method suffers from the necessity to recalculate data centering each time a new point is added to the data matrix.

\subsection{Homogeneous SVD} \label{homog-svd}
There is another approach that allows us to calculate plane parameters without data centering. It was described in \cite{ferrer} as \textit{Homogeneous plane fitting method} and in \cite{normal_est} as \textit{PlaneSVD}.
The following minimization task is solved:
\[
\min_{\bm{b}} \| \left[P^\top \mathbbm{1}_k \right] \bm{b} \|_2,
\]
where $\bm{b}$ stands for $\begin{psmallmatrix}
\bm{n} \\
d
\end{psmallmatrix}$. Similarly to section \ref{center-svd} this task can be solved with SVD of $4 \times 4$ matrix
\[
C = \left[P^\top \mathbbm{1}_k \right]^\top \left[P^\top \mathbbm{1}_k \right].
\]
As soon as the resulting vector $\bm{b}$ is unitary, one should afterward rescale vector $\bm{n}$ to meet the requirement $\|\bm{n}\|_2 = 1$. Oppositely to \textit{Centered SVD} this method supports incremental updates meaning no recentering of data needed; we show how this property can be effectively used in section \ref{integ-im}.


\section{Known normal estimation algorithms}

Several attempts were taken to estimate normals using deep neural network architectures. The current state-of-the-art approach is \cite{deep_surf}, where researchers tried to combine RGB and depth inputs to predict normals. Another interesting work is \cite{adaptive-neighborhood}, where the idea of multiscale input is implemented, but the resulting algorithm is unsupervised.

In our work, we tried to effectively combine known best-practices with modern technologies and create reliable normal estimation algorithm.