В данной работе предлагается новый подход к задаче предсказания нормалей к данным полученным с RGBD сенсора. Этот подход состоит из предсказания нормалей на разных масштабах с использованием простых геометрических методов и последующего объединения полученных нормалей. Объединение производится при помощи полностью сверточной нейронной сети, натренированной над нормалями, восстановленными из нескольких точек обзора. Сравнение этого алгоритма с существующими state-of-the-art подходами показывает, что он достигает почти лучшего возможного качества, несмотря на то что использует только данные о глубине каждого пикселя, не используя RGB каналы. Также предлагается эффективная реализация этого алгоритма, которая превосходит по производительности современные библиотеки для обработки облаков точек.